\documentclass[a4paper,8pt]{article}

\usepackage[left=1.5cm,right=1.5cm,top=1cm,bottom=0.5cm,nohead]{geometry}
\usepackage[T1]{fontenc}
\usepackage[utf8]{inputenc}
\usepackage[francais]{babel}
\usepackage{array}
\usepackage{xcolor}
\usepackage{palatino}
\usepackage[scaled=.92]{helvet}
\pagestyle{empty}
\newcolumntype{P}[1]{>{\raggedleft}p{#1}}
\newcommand{\hsection}[1]{\section*{\fontfamily{phv}\selectfont\textsc{#1}}}
\newcommand{\hsubsection}[1]{\subsection*{\fontfamily{phv}\selectfont\textsc{#1}}}

\begin{document}
\fontfamily{phv}\selectfont
\hsection{Collin Jeremy}
\begin{tabular}{p{16.5cm}}
\hline
32, célibataire\\
111 boulevard Robert Schuman, 44300 Nantes\\
Mobile : +33(0)666647263\\
Email : jeremy.collin.lnk@gmail.com\\
Permis A,B, véhicule\\
\end{tabular}
%\begin{center}
%\textbf{{\Large Architecte infrastructure}}
%\end{center}

\hsubsection{Compétences techniques}
\begin{tabular}{P{3.5cm}|p{13cm}}
\textsl{Hardware}		& PC, Serveurs, Routers (Cisco - notions), Switchs (HP, Cisco)\\
\textsl{Microsoft}		& Windows 9x, XP, 2003 Server, 7, 2008 server R2\\
\textsl{Linux}			& Ubuntu, Debian, RedHat Enterprise Linux, CentOS\\
\textsl{Réseau}			& Dns, dhcp, Samba, nfs, ftp, web\\
\textsl{Sécurité}		& Iptables, vpn\\
\textsl{Programmation}		& Shell (Bash), \LaTeX, html, css\\
\textsl{Mail}			& Exchange, Postfix, Sendmail\\
\textsl{Monitoring}		& Nagios, Centreon\\
\textsl{Cluster}		& Heartbeat, ocfs, RedHat Cluster, VMware ESXi Cluster\\
\textsl{Virtualisation}		& Xen, VMware Workstation, VMware ESXi, Virtualbox\\
\textsl{Base de données}	& MySQL\\
%\textsl{Rédaction}		& Documentation, Procedures\\
\textsl{Gestion}		& Netcenter, incidents techniques, diagnostiques\\
\end{tabular}

\hsubsection{Expérience Professionnelle}
\begin{tabular}{P{3.5cm}|p{13cm}}
Mai 2016 à Juin 2016		& \textbf{Open, Domisys}, Nantes\\
\textsl{Poste}			& DevOps\\
\textsl{Mission}		& Etude de la solution GLPI/OCS actuelle\\
				& Mise en place de tests et recettes pour mise à jour de la solution vers un utilisation optimale  plus complète de l'outil GLPI et Fusioninventory\\
\textsl{Env. Technique}		& Linux, Debian, GLPI, Git, DSI \\
\end{tabular}
\begin{tabular}{P{3.5cm}|p{13cm}}
Janvier 2015 à Mai 2016		& \textbf{Open, DG-MI}, Rennes\\
\textsl{Contexte}      		& DGA Maîtrise de l'information est un centre d'expertise technique de la DGA (Direction Générale de l'Armement), qui a pour mission des études, expertises et essais dans les domaines de la guerre électronique des systèmes d'armes, des systèmes d'information, des télécommunications, de la sécurité de l'information et des composants électronique.\\
\textsl{Poste}			& Gestion technique de Datacenter\\
\textsl{Rôles et activités}	& Administration et gestion de différentes plateformes d'expérimentation\\
				& Maintien en conditions opérationnelles des plateformes\\
				& Intégration de nouvelles plateformes au sein de l'architecture existante\\
				& Evolution des systèmes et réseaux des plateformes\\
				& Migration de plateforme physique vers plateforme virtualisée\\
\textsl{Env. Technique}		& Linux, Windows, Cisco, VMWare ESXi, Defense\\
\end{tabular}
\begin{tabular}{P{3.5cm}|p{13cm}}
Juin 2014 à Décembre 2014	& \textbf{Open, Beaumanoir}, Saint-Malo\\
\textsl{Contexte}		& Groupe de textile, spécialiste de la distribution mode\\
\textsl{Poste}			& Intégration et administration système\\
\textsl{Rôles et activités}	& Installation et maintien en condition opérationnelle de plateformes Linux\\
				& Intégration d'applications packagées sur middleware (Tomcat, Apache)\\
				& Mise en production et évolution des systèmes de monitoring : Nagios/Centreon\\
				& Normalisation du socle système et middleware\\
				& Installation et maintenance opérationnelle des applications internes (dev/pré-prod/prod)\\
				& Installation et maintenance opérationnelle du parc serveur linux (dev/pré-prod/prod)\\
				& Création de scripts d'industrialisation pour tâches récurrentes, installations et diagnostiques\\
				& Conseil sur mise en œuvre de nouvelle infrastructure\\
\textsl{Env. Technique}		& Linux, Debian, VMWare ESXi, Centreon, Bash, OpenLDAP, DHCP(AD, Linux), DNS (AD, BIND), PME\\
\end{tabular}
\begin{tabular}{P{3.5cm}|p{13cm}}
Fevrier 2013 à Septembre 2013	& \textbf{Open, DG-MI}, Rennes\\
\textsl{Poste}			& Gestion technique de Datacenter\\
\textsl{Rôles et activités}	& Administration et gestion de différentes plateformes d'expérimentation\\
				& Maintien en conditions opérationnelles des plateformes\\
				& Intégration de nouvelles plateformes au sein de l'architecture existante\\
				& Evolution des systèmes et réseaux des plateformes\\
				& Migration de plateforme physique vers plateforme virtualisée\\
\textsl{Env. Technique}		& Linux, Windows, Cisco, VMWare ESXi, Defense\\
\end{tabular}
\begin{tabular}{P{3.5cm}|p{13cm}}
Mai 2012 à Novembre 2012	& \textbf{Open, SNCF}, Lyon\\
\textsl{Poste}			& Ingénierie système socle\\
\textsl{Projets}		& Mise en place de script d'industrialisation pour installation et tâches récurrentes\\
				& Création de script de diagnostic et de traitement des données\\
\textsl{Rôles et activités}	& Installation et maintien en condition opérationnelle de plateformes Linux\\
				& Mise à disposition pour les clients de plateformes opérationnelles\\
				& Maintiens et évolution du socle système et applicatif\\
				& Diagnostics et résolution de problèmes machines et systèmes\\
				& Collaboration avec différents services (Middleware, BDD, exploitation) pour le suivi et la validation des machines\\
\textsl{Env. Technique}		& Redhat, SuSE, Debian, scripting, bash, ksh, awk, sed, grand compte\\
\end{tabular}
\begin{tabular}{P{3.5cm}|p{13cm}}
Novembre 2011 à Mai 2012	& \textbf{Open, SNCF}, Lyon\\
\textsl{Poste}			& Administration système\\
\textsl{Rôles et activités} 	& Administration système Linux Redhat, Windows Server 2003\\
				& Gestion des serveurs DNS (Linux et Windows 2003) et DHCP\\
				& Déploiement d'applications packagées\\
				& Diagnostics et résolutions de problèmes liés aux plateformes d'hébergement des sites client\\
				& Maintien en condition opérationnelle des applications métiers\\
\textsl{Env. Technique}		& Redhat, Linux, Windows Server 2003, DNS, BIND, Apache, grand compte\\
\end{tabular}
\begin{tabular}{P{3.5cm}|p{13cm}}
Mai 2011 à Octobre 2011		& \textbf{Open, Cégélec}, Lyon\\
\textsl{Contexte}		& Les entreprises Cegelec conçoivent, installent et maintiennent des systèmes dans l’industrie, les infrastructures et le tertiaire, particulièrement dans des secteurs à forte demande comme par exemple l’énergie, le pétrole, le bâtiment et les travaux publics.\\
\textsl{Poste}			& Ingénierie système et réseaux\\
\textbf{Projet:~}\textit{Stella} & Création de l'infrastructure complète (système et réseau)\\
				& Installation Cluster VMWare ESXi\\
				& Installation et intégration de stockage SAN (HP MSA) pour le Cluster ESXi\\
				& Intégration de l'environnement réseau HP Procurve\\
				& Installation et mise à disposition de VMs Windows Server 2003\\
				& Gestion logistique du matériel (mise en baie et répartition homogène)\\
				& Intégration des applicatifs sur clients lourds\\
\textbf{Projet:~}\textit{DorBreizh}	& Intégration d'applicatifs clients (JBOSS, Tomcat)\\
				& Intégration de Cluster Linux (HeartBeat/RedHat) en environnement de production\\
				& Création de scripts d'installation et déploiement automatisés ou interactifs\\
				& Migration de serveurs et applicatifs en environnement de production\\
				& Tests préparatoires de migration et d'automatisation\\
\textbf{Projet:~}\textit{Gentiane}	& Intégration Cluster Linux (HeartBeat/RedHat)\\
				& Mise en œuvre du stockage SAN avec accès concurrent pour le Cluster Linux (OCFS)\\
\textbf{Projet:~}\textit{Vauban} & Tests de migration d'OS Linux 32bits vers 64bits\\
				& Préparation des outils de migration (disques, software)\\
				& Migration de plateforme applicative en environnement de production\\
				& Diagnostics et résolution de problèmes post-migration\\
\textbf{Projet:~}\textit{Itinisère}	& Définition de l’Architecture de l'infrastructure du projet\\
				& Dimensionnement et validation des infrastructures systèmes et cluster ESXi, de l'infrastructure réseau LAN et MAN, et du dimensionnement de la solution de stockage SAN\\
				& Création des protocoles d'installation et d'intégration\\
				& Design général et détaillé de l'infrastructure\\
\textsl{Rôles et activités}	& Architecture et intégration système, réseau et applicative\\
\textsl{Environnement technique} & VMware ESXi, Windows Server 2008, Redhat, CentOS, Cluster, VSphere, Stockage, Apache, Centreon\\
\end{tabular}
\begin{tabular}{P{3.5cm}|p{13cm}}
Juillet 2010 à Mai 2011	& \textbf{Helice, SFR}, Lyon\\
\textsl{Poste}			& Gestion technique de Datacenter\\
\textsl{Mission}		& Installations, survey, management de salles, interventions d'urgence (astreintes)\\
\textsl{Env. Technique}		& Datacenter\\
\end{tabular}
\begin{tabular}{P{3.5cm}|p{13cm}}
Février à Juillet 2010	& \textbf{Axians, cdd-interim}, Lyon\\
\textsl{Poste}	 		& Technicien réseaux\\
\textsl{Mission}		& Migrations switchs et routeurs en environement de production, installation ToIP\\
\textsl{Env. Technique}		& HP Procurve, Routeurs Cisco\\
\end{tabular}
\begin{tabular}{P{3.5cm}|p{13cm}}
Septembre à Décembre 2009	& \textbf{Institut de Physique Nucleaire de Lyon (UMR CNRS-UCBL)}, Lyon\\
\textsl{Poste}	 		& Support utilisateur Linux (vacataire)\\
\textsl{Mission} 		& Installation, maintenance, réparation postes utilisateurs\\
\textsl{Technique Env.}		& Scientific Linux, RedHat Enterprise Linux, Ubuntu Linux\\
\textsl{Réalisations}		& \textsl{\'{E}tude de l'architecture d'un SunFire X4500 pour utilisation en ferme de calcul}\\
\end{tabular}
\begin{tabular}{P{3.5cm}|p{13cm}}
Septembre 2007 à Juillet 2009	& \textbf{Medimex, alternance}, Lyon\\
\textsl{Contexte}		& Spécialiste de l’évaluation fonctionnelle et de la rééducation.\\
\textsl{Projet}			& Refonte complète de l'infrastructure système et réseau de l'entreprise\\
\textsl{Rôles et activités} 	& Gestion de l'ensemble du parc machines (serveurs, postes client)\\
				& Administration du réseau informatique et téléphonique sur IP\\
				& Mise à jour, maintien en condition opérationnelle et évolution des serveurs\\
				& Mise en place de politiques de sécurité\\
				& Mise en place de politiques de sauvegarde\\
				& Mise en place de plan de reprise d'activité\\
				& Création de scripts d'industrialisation et d'installation\\
				& Rédaction de documentation exhaustive sur l'infrastructure du système d'information\\
				& Mise à l'épreuve des politiques de sauvegarde et de reprise d'activité\\
				& Formation et support aux utilisateurs\\
\textsl{Env. technique}		& Windows, Linux Debian, Xen, Virtualisation, sécurité, Apache\\
\end{tabular}

\hsubsection{Scolaire}
\begin{tabular}{P{3.5cm}|p{13cm}}
Depuis Sept. 2010		& \textbf{Formation ingénieur systèmes et réseaux}\\
	& \textsc{CNAM - Rhône-Alpes}\\
 & \\
2007 -- 2009 			& \textbf{Technicien Supérieur en Maintenance Informatique et Réseaux}\\
	& \textsc{Ciefa} Lyon\\
\end{tabular}

\hsubsection{Compétences linguistiques}
\begin{tabular}{P{3.5cm}|p{13cm}}
\textsl{Anglais}		& Courant, technique\\
\textsl{Allemand}		& Scolaire\\
\textsl{Japonais}		& Notions\\
\end{tabular}

\hsubsection{Loisirs}
\begin{tabular}{P{3.5cm}|p{13cm}}
\textsl{Divers}			& Photographie, jeux-videos, lecture\\
\textsl{Sports}			& Marche, Roller\\
\end{tabular}

\end{document}
